\chapter{Arhitektura i dizajn sustava}
		
		\textbf{\textit{dio 1. revizije}}\\



               \textit{Arhitektura sustava može se podijeliti  na tri podsustava: }

              \begin{packed_item}
               \item  Web  preglednik
                \item   Web aplikacija
                \item    Baza Podataka

                \end{packed_item}

                 \begin{figure}[H]
                     \includegraphics[width=7cm, height=5cm ]{arhitektura.jpg}
                      \centering
                      \caption{Arhitektura sustava}
                  \end{figure}


                  \textbf{Web preglednik} je aplikacija za prikaz web stranica. Kada korisnik posjeti neku web stranicu, internetski preglednik zatraži podatke
                                 sa web poslužitelja, koje interpretira i prikaže na zaslon računala ili nekog drugog pametnog uređaja.

                  \textbf{Web poslužitelj} osnovni podsustav u izradi web aplikacije. Njegova uloga u sustavu je komunikacija korisnika(web preglednika) s 
                                 aplikacijom. Komunikacija se ostvaruje preko HTTP (engl. Hyper Text Transfer Protocol) protokola. Uloga HTTP protokola je prijenos 
                                 sadržaja s poslužitelja na preglednik, koji dalje odrađuje svoj posao.
                            
                                Korisnik preko preglednika, koristi korisničko sučelje, te tako šalje HTTP zahtjeve na poslužitelj. Neki od zahtjeva su HTTP GET, HTTP POST 
                                itd. Svaki takav zahtjev u 99 \% slučajeva na poslužitelju izaziva, njegovu komunikaciju s \underbar{bazom podataka}.

                  \textbf{Baza podataka} je skup međusobno povezanih podataka, pohranjeni u vanjskoj memoriji računala. Njezina uloga u sustavu je brza i efikasna
                                 pohrana podataka, koji se propagiraju iz sloja web preglednika, preko web poslužitelja.

                  \textit{Web aplikacija se najčešće dijeli na dva dijela: }

                               \begin{packed_item}
              			 \item  \textbf{Front-end} koji je zadužen za razvoj korsicničkog iskustva na webu.
                		\item    \textbf{Back-end} koji je zadužen za obradu i spremanje podataka, koji dođu sa frontenda.

              		       \end{packed_item}

                                Tehnologije koje smo uzeli za razvoj web aplikacije su React.js za Front-end u razvojnom okruženju Microsoft Visual Studio Code, te .NET za Back-end u razvojnom okruženju Microsoft Visual Studio.
                                Uz .Net na back-endu, smo uzeli SQL server za bazu podataka. 
                                Arhitektura sustava će biti bazirana na MVC(engl. Model-View-Controller) konceptu.

                      \textbf{MVC} model sastoji se od tri komponenti: 
                                  
                               \begin{packed_item}
             			  \item \textbf{Model} -  komponenta modela koja je zadužena za dohvat i manipulacijom podataka. Često za obavljanje svojih zadaća koristi bazu podataka.
                		  \item  \textbf{View} -   komponenta kojoj je zadaća prikaz dobivenih podataka korisniku.
               		           \item  \textbf{Controller} - komponenta zadužena za primanje zahtjeva od korisnika, koje dalje propagira komponenti Model.

                                \end{packed_item}
                  
                    \begin{figure}[H]
                     \includegraphics[width=7cm, height=5cm ]{mvc.jpg}
                      \centering
                      \caption{MVC model}
                    \end{figure}
                    
                                 

			
		\section{Baza podataka}
			
			\textbf{\textit{dio 1. revizije}}\\
			
		\textit{Potrebno je opisati koju vrstu i implementaciju baze podataka ste odabrali, glavne komponente od kojih se sastoji i slično.}
		
			\subsection{Opis tablica}
			

				\textit{Svaku tablicu je potrebno opisati po zadanom predlošku. Lijevo se nalazi točno ime varijable u bazi podataka, u sredini se nalazi tip podataka, a desno se nalazi opis varijable. Svjetlozelenom bojom označite primarni ključ. Svjetlo plavom označite strani ključ}
				
				
				\begin{longtblr}[
					label=none,
					entry=none
					]{
						width = \textwidth,
						colspec={|X[6,l]|X[6, l]|X[20, l]|}, 
						rowhead = 1,
					} %definicija širine tablice, širine stupaca, poravnanje i broja redaka naslova tablice
					\hline \multicolumn{3}{|c|}{\textbf{korisnik - ime tablice}}	 \\ \hline[3pt]
					\SetCell{LightGreen}IDKorisnik & INT	&  	Lorem ipsum dolor sit amet, consectetur adipiscing elit, sed do eiusmod  	\\ \hline
					korisnickoIme	& VARCHAR &   	\\ \hline 
					email & VARCHAR &   \\ \hline 
					ime & VARCHAR	&  		\\ \hline 
					\SetCell{LightBlue} primjer	& VARCHAR &   	\\ \hline 
				\end{longtblr}
				
				
			
			\subsection{Dijagram baze podataka}
				\textit{ U ovom potpoglavlju potrebno je umetnuti dijagram baze podataka. Primarni i strani ključevi moraju biti označeni, a tablice povezane. Bazu podataka je potrebno normalizirati. Podsjetite se kolegija "Baze podataka".}
			
			\eject
			
			
		\section{Dijagram razreda}
		
			\textit{Potrebno je priložiti dijagram razreda s pripadajućim opisom. Zbog preglednosti je moguće dijagram razlomiti na više njih, ali moraju biti grupirani prema sličnim razinama apstrakcije i srodnim funkcionalnostima.}\\
			
			\textbf{\textit{dio 1. revizije}}\\
			
			\textit{Prilikom prve predaje projekta, potrebno je priložiti potpuno razrađen dijagram razreda vezan uz \textbf{generičku funkcionalnost} sustava. Ostale funkcionalnosti trebaju biti idejno razrađene u dijagramu sa sljedećim komponentama: nazivi razreda, nazivi metoda i vrste pristupa metodama (npr. javni, zaštićeni), nazivi atributa razreda, veze i odnosi između razreda.}\\
			
			\textbf{\textit{dio 2. revizije}}\\			
			
			\textit{Prilikom druge predaje projekta dijagram razreda i opisi moraju odgovarati stvarnom stanju implementacije}
			
			
			
			\eject
		
		\section{Dijagram stanja}
			
			
			\textbf{\textit{dio 2. revizije}}\\
			
			\textit{Potrebno je priložiti dijagram stanja i opisati ga. Dovoljan je jedan dijagram stanja koji prikazuje \textbf{značajan dio funkcionalnosti} sustava. Na primjer, stanja korisničkog sučelja i tijek korištenja neke ključne funkcionalnosti jesu značajan dio sustava, a registracija i prijava nisu. }
			
			
			\eject 
		
		\section{Dijagram aktivnosti}
			
			\textbf{\textit{dio 2. revizije}}\\
			
			 \textit{Potrebno je priložiti dijagram aktivnosti s pripadajućim opisom. Dijagram aktivnosti treba prikazivati značajan dio sustava.}
			
			\eject
		\section{Dijagram komponenti}
		
			\textbf{\textit{dio 2. revizije}}\\
		
			 \textit{Potrebno je priložiti dijagram komponenti s pripadajućim opisom. Dijagram komponenti treba prikazivati strukturu cijele aplikacije.}