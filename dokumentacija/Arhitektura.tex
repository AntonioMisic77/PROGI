\chapter{Arhitektura i dizajn sustava}
		
		\textbf{\textit{dio 1. revizije}}\\

		\textit{ Potrebno je opisati stil arhitekture te identificirati: podsustave, preslikavanje na radnu platformu, spremišta podataka, mrežne protokole, globalni upravljački tok i sklopovsko-programske zahtjeve. Po točkama razraditi i popratiti odgovarajućim skicama:}
	\begin{itemize}
		\item 	\textit{izbor arhitekture temeljem principa oblikovanja pokazanih na predavanjima (objasniti zašto ste baš odabrali takvu arhitekturu)}
		\item 	\textit{organizaciju sustava s najviše razine apstrakcije (npr. klijent-poslužitelj, baza podataka, datotečni sustav, grafičko sučelje)}
		\item 	\textit{organizaciju aplikacije (npr. slojevi frontend i backend, MVC arhitektura) }		
	\end{itemize}

	
		

		

				
		\section{Baza podataka}
			
			\textbf{\textit{dio 1. revizije}}\\
			
		\textit{Potrebno je opisati koju vrstu i implementaciju baze podataka ste odabrali, glavne komponente od kojih se sastoji i slično.}
		
			\subsection{Opis tablica}
			

				\textit{Svaku tablicu je potrebno opisati po zadanom predlošku. Lijevo se nalazi točno ime varijable u bazi podataka, u sredini se nalazi tip podataka, a desno se nalazi opis varijable. Svjetlozelenom bojom označite primarni ključ. Svjetlo plavom označite strani ključ}
				

\textbf{Users} Ovaj entitet sadrži sve važne informacije o korisniku aplikacije. Sadrži atribute: korisničko ime, OIB, ime, prezime, fotografiju, broj telefona, adresu elektroničke pošte, lozinku, ID uloge korisnika te status je li potvrđen. Ovaj entitet u vezi je \textit{Many-to-One} s entitetom Role preko atributa identifikatora uloge, u vezi \textit{One-To-Many} s entitetom Operation preko atributa identifikatora osobe, u vezi \textit{One-to-Many} s entitetom Comment preko atributa identifikatora osobe, u vezi \textit{One-To-Many} s entitetom Area preko atributa identifikatora osobe te u vezi \textit{One-To-Many} s entitetom Block preko atributa identifikatora osobe. 
				\begin{longtblr}[
					label=none,
					entry=none
					]{
						width = \textwidth,
						colspec={|X[6,l]|X[6, l]|X[20, l]|}, 
						rowhead = 1,
					} %definicija širine tablice, širine stupaca, poravnanje i broja redaka naslova tablice
					\hline \multicolumn{3}{|c|}{\textbf{Users}}	 \\ \hline[3pt]
					Username & VARCHAR	&  	korisničko ime korisnika  	\\ \hline
					\SetCell{LightGreen}OIB	& BIGINT & jedinstveni identifikator osobe   	\\ \hline 
					FirstName & VARCHAR & ime korisnika  \\ \hline 
					LastName & VARCHAR	& prezime korisnika 		\\ \hline
					Photo & VARCHAR & fotografija korisnika \\ \hline 
				    PhoneNumber & VARCHAR & telefonski broj korisnika  	\\ \hline 
					EMail & VARCHAR & adresa elektroničke pošte korisnika  	\\ \hline
					Password & VARCHAR & hash lozinke  	\\ \hline
                    \SetCell{LightBlue}RoleId & INT & identifikator uloge  	\\ \hline
                    Confirmed & BIT & status je li administrator potvrdio korisnika  	\\ \hline
				\end{longtblr}
				
\textbf{Role} Ovaj entitet sadrži sve važne informacije o ulogama korisnika aplikacije. Sadrži atribute ID uloge i ime uloge. Ovaj entitet u vezi je \textit{One-to-Many} s entitetom Users preko atributa identifikatora uloge. 
				\begin{longtblr}[
					label=none,
					entry=none
					]{
						width = \textwidth,
						colspec={|X[6,l]|X[6, l]|X[20, l]|}, 
						rowhead = 1,
					} %definicija širine tablice, širine stupaca, poravnanje i broja redaka naslova tablice
					\hline \multicolumn{3}{|c|}{\textbf{Role}}	 \\ \hline[3pt]
					\SetCell{LightGreen}Id & INT	&  jedinstveni identifikator uloge  	\\ \hline
					Name	& VARCHAR &  naziv uloge 	\\ \hline
				\end{longtblr}
			
\textbf{Missing Report} Ovaj entitet sadrži sve važne informacije o prijavama nestalih osoba. Sadrži atribute: ID prijave, ime osobe, prezime osobe, OIB osobe, fotografiju osobe, opis prijave, datum i vrijeme prijave te datum i vrijeme pronalaska. Ovaj entitet u vezi je \textit{One-to-Many} s entitetom Comment preko atributa identifikatora prijave nestale osobe.  
				\begin{longtblr}[
					label=none,
					entry=none
					]{
						width = \textwidth,
						colspec={|X[6,l]|X[6, l]|X[20, l]|}, 
						rowhead = 1,
					} %definicija širine tablice, širine stupaca, poravnanje i broja redaka naslova tablice
					\hline \multicolumn{3}{|c|}{\textbf{Missing Report}}	 \\ \hline[3pt]
					\SetCell{LightGreen}Id & INT	&  jedinstveni identifikator prijave nestale osobe  	\\ \hline
					FirstName	& VARCHAR &  ime nestale osobe 	\\ \hline
                    LastName	& VARCHAR &  prezime nestale osobe 	\\ \hline
                    OIB	& BIGINT &  jedinstveni identifikator nestale osobe 	\\ \hline
                    Photo	& VARCHAR &  fotografija nestale osobe 	\\ \hline
                    Description	& VARCHAR &  opis prijave nestale osobe 	\\ \hline
                    ReportedAt & DATETIME & datum i vrijeme prijave nestanka osobe   \\ \hline
                    FoundAt & DATETIME & datum i vrijeme pronalaska nestale osobe \\ \hline
				\end{longtblr}
			
\textbf{Comment} Ovaj entitet sadrži sve važne informacije o komentarima prijava nestalih osoba. Sadrži atribute ID komentara, ID prijave koja se komentira, sadržaj komentara, OIB korisnika. Ovaj entitet u vezi je \textit{Many-to-One} s entitetom Missing Report preko atributa identifikatora prijave nestale osobe i u vezi \textit{Many-to-One} s entitetom Users preko atributa identifikatora osobe. 
				\begin{longtblr}[
					label=none,
					entry=none
					]{
						width = \textwidth,
						colspec={|X[6,l]|X[6, l]|X[20, l]|}, 
						rowhead = 1,
					} %definicija širine tablice, širine stupaca, poravnanje i broja redaka naslova tablice
					\hline \multicolumn{3}{|c|}{\textbf{Comment}}	 \\ \hline[3pt]
					\SetCell{LightGreen}Id & INT	&  jedinstveni identifikator komentara  	\\ \hline
					\SetCell{LightBlue}ReportId	& INT &  jedinstveni identifikator prijave nestale osobe 	\\ \hline
                    Text & VARCHAR &  sadržaj komentara 	\\ \hline
                    \SetCell{LightBlue} UserOIB	& BIGINT &  jedinstveni identifikator osobe koja komentira 	\\ \hline
				\end{longtblr}

\textbf{Operation} Ovaj entitet sadrži sve važne informacije o operacijama. Sadrži atribute: ID operacije, status operacije i OIB voditelja operacije. Ovaj entitet u vezi je \textit{Many-to-One} s entitetom Users preko atributa identifikatora osobe te u vezi \textit{One-To-Many} s entitetom Region preko atributa identifikatora operacije. 
				\begin{longtblr}[
					label=none,
					entry=none
					]{
						width = \textwidth,
						colspec={|X[6,l]|X[6, l]|X[20, l]|}, 
						rowhead = 1,
					} %definicija širine tablice, širine stupaca, poravnanje i broja redaka naslova tablice
					\hline \multicolumn{3}{|c|}{\textbf{Operation}}	 \\ \hline[3pt]
					\SetCell{LightGreen}Id & INT	&  jedinstveni identifikator operacije\\ \hline
					Status	& VARCHAR &  status operacije 	\\ \hline
                    \SetCell{LightBlue}LeaderOIB	& BIGINT & jedinstveni identifikator voditelja operacije 	\\ \hline
				\end{longtblr}

\textbf{Area} Ovaj entitet sadrži sve važne informacije o područjima. Sadrži atribute: identifikator područja, datum i vrijeme nastanka područja, datum i vrijeme zatvaranja područja te identifikator osobe koja je zadnja uređivala područje. Ovaj entitet u vezi je \textit{Many-to-One} s entitetom Users preko atributa identifikatora osobe, u vezi \textit{One-To-Many} s entitetom Block preko atributa identifikatora područja, u vezi \textit{One-To-Many} s entitetom Building preko atributa identifikatora područja, u vezi \textit{One-To-Many} s entitetom Point preko atributa identifikatora područja te u vezi \textit{One-To-Many} s entitetom Region preko atributa identifikatora područja. 
				\begin{longtblr}[
					label=none,
					entry=none
					]{
						width = \textwidth,
						colspec={|X[6,l]|X[6, l]|X[20, l]|}, 
						rowhead = 1,
					} %definicija širine tablice, širine stupaca, poravnanje i broja redaka naslova tablice
					\hline \multicolumn{3}{|c|}{\textbf{Area}}	 \\ \hline[3pt]
					\SetCell{LightGreen}Id & INT	&  jedinstveni identifikator područja \\ \hline
					CreatedAt & DATETIME & datum i vrijeme nastanka područja \\ \hline
					ClosedAt & DATETIME & datum i vrijeme zatvaranja područja \\ \hline
					\SetCell{LightBlue}UpdatedLastByOIB	& BIGINT &  identifikator osobe koja je zadnja uređivala područje 	\\ \hline
				\end{longtblr}  
                
\textbf{Region} Ovaj entitet sadrži sve važne informacije o regiji. Sadrži atribute: identifikator regije te identifikator operacije. Ovaj entitet u vezi je \textit{Many-to-One} s entitetom Area preko atributa identifikatora regije, u vezi \textit{Many-To-One} s entitetom Operation preko atributa identifikatora operacije te u vezi \textit{One-To-Many} s entitetom Block preko atributa identifikatora regije. 
				\begin{longtblr}[
					label=none,
					entry=none
					]{
						width = \textwidth,
						colspec={|X[6,l]|X[6, l]|X[20, l]|}, 
						rowhead = 1,
					} %definicija širine tablice, širine stupaca, poravnanje i broja redaka naslova tablice
					\hline \multicolumn{3}{|c|}{\textbf{Region}}	 \\ \hline[3pt]
					\SetCell{LightGreen}AreaId & INT	&  jedinstveni identifikator regije \\ \hline
					\SetCell{LightBlue}OperationId	& INT &  jedinstveni identifikator operacije 	\\ \hline
				\end{longtblr}            
				
\textbf{Block} Ovaj entitet sadrži sve važne informacije o blokovima. Sadrži atribute: identifikator bloka, status bloka, identifikator regije te identifikator osobe koja uređuje blok. Ovaj entitet u vezi je \textit{Many-to-One} s entitetom Users preko atributa identifikatora osobe, u vezi \textit{Many-To-One} s entitetom Area preko atributa identifikatora bloka, u vezi \textit{Many-To-One} s entitetom Region preko atributa identifikatora regije te u vezi \textit{One-To-Many} s entitetom Building preko atributa identifikatora bloka. 
	          \begin{longtblr}[
	          	label=none,
	          	entry=none
	          	]{
	          		width = \textwidth,
	          		colspec={|X[6,l]|X[6, l]|X[20, l]|}, 
	          		rowhead = 1,
	          	} %definicija širine tablice, širine stupaca, poravnanje i broja redaka naslova tablice
	          	\hline \multicolumn{3}{|c|}{\textbf{Block}}	 \\ \hline[3pt]
	          	\SetCell{LightGreen}AreaId & INT	&  jedinstveni identifikator bloka \\ \hline
	          	Status & VARCHAR & status bloka \\ \hline
	          	\SetCell{LightBlue}RegionId & INT & jedinstveni identifikator regije \\ \hline
	          	\SetCell{LightBlue}ActiveForOIB	& BIGINT &  identifikator kartografa koji trenutno uređuje blok 	\\ \hline
	          \end{longtblr}  
				
\textbf{Building} Ovaj entitet sadrži sve važne informacije o građevinama. Sadrži atribute: identifikator građevine, identifikator bloka te status građevine. Ovaj entitet u vezi je \textit{Many-to-One} s entitetom Area preko atributa identifikatora građevine te u vezi \textit{Many-To-One} s entitetom Block preko atributa identifikatora bloka.
			\begin{longtblr}[
				label=none,
				entry=none
				]{
					width = \textwidth,
					colspec={|X[6,l]|X[6, l]|X[20, l]|}, 
					rowhead = 1,
				} %definicija širine tablice, širine stupaca, poravnanje i broja redaka naslova tablice
				\hline \multicolumn{3}{|c|}{\textbf{Building}}	 \\ \hline[3pt]
				\SetCell{LightGreen}AreaId & INT	&  jedinstveni identifikator građevine \\ \hline
				\SetCell{LightBlue}BlockId & INT & jedinstveni identifikator bloka \\ \hline
	          	Status & VARCHAR & status građevine \\ \hline
			\end{longtblr}

\textbf{Point} Ovaj entitet sadrži sve važne informacije o točkama i njihovim koordinatama. Sadrži atribute: identifikator točke, geografska širina, geografska dužina, identifikator područja te redni broj. Ovaj entitet u vezi je \textit{Many-to-One} s entitetom Area preko atributa identifikatora područja.
			\begin{longtblr}[
				label=none,
				entry=none
				]{
					width = \textwidth,
					colspec={|X[6,l]|X[6, l]|X[20, l]|}, 
					rowhead = 1,
				} %definicija širine tablice, širine stupaca, poravnanje i broja redaka naslova tablice
				\hline \multicolumn{3}{|c|}{\textbf{Point}}	 \\ \hline[3pt]
				\SetCell{LightGreen}Id & INT	&  jedinstveni identifikator točke \\ \hline
				Latitude & FLOAT & geografska širina točke \\ \hline
				Longitude & FLOAT & geografska dužina točke \\ \hline
				\SetCell{LightBlue}AreaId & INT & jedinstveni identifikator područja \\ \hline
				OrderNumber & INT & redni broj točke \\ \hline
			\end{longtblr}
			
			\subsection{Dijagram baze podataka}
				\textit{ U ovom potpoglavlju potrebno je umetnuti dijagram baze podataka. Primarni i strani ključevi moraju biti označeni, a tablice povezane. Bazu podataka je potrebno normalizirati. Podsjetite se kolegija "Baze podataka".}
			
			\eject
			
			
		\section{Dijagram razreda}
		
			\textit{Potrebno je priložiti dijagram razreda s pripadajućim opisom. Zbog preglednosti je moguće dijagram razlomiti na više njih, ali moraju biti grupirani prema sličnim razinama apstrakcije i srodnim funkcionalnostima.}\\
			
			\textbf{\textit{dio 1. revizije}}\\
			
			\textit{Prilikom prve predaje projekta, potrebno je priložiti potpuno razrađen dijagram razreda vezan uz \textbf{generičku funkcionalnost} sustava. Ostale funkcionalnosti trebaju biti idejno razrađene u dijagramu sa sljedećim komponentama: nazivi razreda, nazivi metoda i vrste pristupa metodama (npr. javni, zaštićeni), nazivi atributa razreda, veze i odnosi između razreda.}\\
			
			\textbf{\textit{dio 2. revizije}}\\			
			
			\textit{Prilikom druge predaje projekta dijagram razreda i opisi moraju odgovarati stvarnom stanju implementacije}
			
			
			
			\eject
		
		\section{Dijagram stanja}
			
			
			\textbf{\textit{dio 2. revizije}}\\
			
			\textit{Potrebno je priložiti dijagram stanja i opisati ga. Dovoljan je jedan dijagram stanja koji prikazuje \textbf{značajan dio funkcionalnosti} sustava. Na primjer, stanja korisničkog sučelja i tijek korištenja neke ključne funkcionalnosti jesu značajan dio sustava, a registracija i prijava nisu. }
			
			
			\eject 
		
		\section{Dijagram aktivnosti}
			
			\textbf{\textit{dio 2. revizije}}\\
			
			 \textit{Potrebno je priložiti dijagram aktivnosti s pripadajućim opisom. Dijagram aktivnosti treba prikazivati značajan dio sustava.}
			
			\eject
		\section{Dijagram komponenti}
		
			\textbf{\textit{dio 2. revizije}}\\
		
			 \textit{Potrebno je priložiti dijagram komponenti s pripadajućim opisom. Dijagram komponenti treba prikazivati strukturu cijele aplikacije.}
