\chapter{Specifikacija programske potpore}
		
	\section{Funkcionalni zahtjevi}
			
			\textbf{\textit{dio 1. revizije}}\\
			
			\textit{Navesti \textbf{dionike} koji imaju \textbf{interes u ovom sustavu} ili  \textbf{su nositelji odgovornosti}. To su prije svega korisnici, ali i administratori sustava, naručitelji, razvojni tim.}\\
				
			\textit{Navesti \textbf{aktore} koji izravno \textbf{koriste} ili \textbf{komuniciraju sa sustavom}. Oni mogu imati inicijatorsku ulogu, tj. započinju određene procese u sustavu ili samo sudioničku ulogu, tj. obavljaju određeni posao. Za svakog aktora navesti funkcionalne zahtjeve koji se na njega odnose.}\\
			
			
			\noindent \textbf{Dionici:}
			
			\begin{packed_enum}
				
				\item Neregistrirani korisnik
				\item Voditelj operacije				
				\item Kartograf
				\item Spasioc
				\item Administrator
				
			\end{packed_enum}
			
			\noindent \textbf{Aktori i njihovi funkcionalni zahtjevi:}
			
			
			\begin{packed_enum}
				\item  \underbar{Neregistrirani/neprijavljeni korisnik može:}
				
				\begin{packed_enum}
					
					\item poslati zahtjev za registraciju s potrebim osobnim podacima: korisničko ime, fotografija, ime, prezime, lozinka, broj mobitela i adresa, i s ulogom za koju se prijavljuje
					\item prijaviti nestalu osobu s imenom, prezimenom, fotografijom i opisom
					
					
				\end{packed_enum}
			
				\item  \underbar{Voditelj operacije može:}
				
				\begin{packed_enum}
					
					\item zaključati prijavu o nestaloj osobi nakon što osoba bude pronađena
					\item obrisati prijavu o nestaloj osobi i komentare na prijavi
					\item definirati područja na karti koja je potrebno pretražiti i kartografirati
					\item imati pristup statistici humanitarne akcije
					\item mijenjati podatke u svom profilu

					
				\end{packed_enum}
				
				\item  \underbar{Kartograf može:}
				
				\begin{packed_enum}
					
					\item promijeniti status odabranog bloka na karti iz nezapočetog u aktivno, i iz aktivnog u provjeru
					\item promijeniti status odabranog bloka na karti iz provjera u aktivno ako posao nije dobro odrađen, te nastavljati uređivati područje
					\item dodavati nove građevine bloku sa sastusom aktivno
					\item imati pristup statistici humanitarne akcije
					\item mijenjati podatke u svom profilu


					
				\end{packed_enum}
				
				\item  \underbar{Spasioc može:}
				
				\begin{packed_enum}
					
					\item pregledavati kartu sa unesenim građevinama i promijeniti im status u pretraženo ili nepretraženo
					\item zaključati prijavu o nestaloj osobi nakon što osoba bude pronađena
					\item mijenjati podatke u svom profilu

				
				\end{packed_enum}
				
				\item  \underbar{Administrator može:}
				
				\begin{packed_enum}
					
					\item potvrditi željena prava neregistriranog koristnika
					\item vidjeti popis registriranih korisnika i njihovih osobnih podataka
					\item mijenjati dodijeljena prava registriranih korisnika
					\item imati sva prava i ovlasti kao registrirani korisnici

				
				\end{packed_enum}
				
				
				
			\end{packed_enum}
			
			
			\eject 
			
			
				
			\subsection{Obrasci uporabe}
				
				\textbf{\textit{dio 1. revizije}}
				
				\subsubsection{Opis obrazaca uporabe}
					\textit{Funkcionalne zahtjeve razraditi u obliku obrazaca uporabe. Svaki obrazac je potrebno razraditi prema donjem predlošku. Ukoliko u nekom koraku može doći do odstupanja, potrebno je to odstupanje opisati i po mogućnosti ponuditi rješenje kojim bi se tijek obrasca vratio na osnovni tijek.}\\
					

					
					\noindent \underbar{\textbf{UC1 - Registarcija u sustav}}
					\begin{packed_item}
	
						\item \textbf{Glavni sudionik: }Korisnik
						\item  \textbf{Cilj:} Registracija u sustavu
						\item  \textbf{Sudionici:} Baza podataka
						\item  \textbf{Preduvjet:} -
						\item  \textbf{Opis osnovnog tijeka:}
						
						\item[] \begin{packed_enum}
	
							\item Korisnik odabire opciju za registraciju u sustav
							\item Korisnik upisuje tražene osobne podatke i ulogu za koju se prijavljuje
							\item Korisnik prima obavijest o uspješno obavljenoj registraciji
						\end{packed_enum}
						
						\item  \textbf{Opis mogućih odstupanja:}
						
						\item[] \begin{packed_item}
	
							\item[2.a] Korisnik odabire već postojeće korisničko ime i/ili e-mail, ili unosi podatke u nedozvoljenom (neispravnom) formatu
							\item[] \begin{packed_enum}
								
								\item Sustav onemogućuje registraciju, te vraća korisnika na stranicu za registraciju
								\item Korisnik ispravlja pogrešno upisane podatke te registracija završava uspješno, ili odustaje od registracije
								
							\end{packed_enum}
							
							
						\end{packed_item}
					\end{packed_item}
					
					\noindent \underbar{\textbf{UC2 - Prijava u sustav}}
					\begin{packed_item}
	
						\item \textbf{Glavni sudionik: }Korisnik
						\item  \textbf{Cilj:}Stjecanje odgovarajućih ovlasti u aplikaciji
						\item  \textbf{Sudionici:} Baza podataka
						\item  \textbf{Preduvjet:} Uspješna registracija
						\item  \textbf{Opis osnovnog tijeka:}
						
						\item[] \begin{packed_enum}
	
							\item Korisnik unosi korisničko ime i lozinku
							\item Potvrda o ispravnosti unesenih podataka
							\item Dobivanje odgovarajućih ovlasti u aplikaciji
						\end{packed_enum}
						
						\item  \textbf{Opis mogućih odstupanja:}
						
						\item[] \begin{packed_item}
	
							\item[2.a] Neispravnost unesenog korisničkog imena i/ili lozinke
							\item[] \begin{packed_enum}
								
								\item Sustav obavještava korisnika o neuspjeloj prijavi u sustav i vraća ga na homepage
								
								
							\end{packed_enum}
							
						\end{packed_item}
					\end{packed_item}
					
				    \noindent \underbar{\textbf{UC3 - Pregled nestalih osoba}}
					\begin{packed_item}
	
						\item \textbf{Glavni sudionik: }Korisnik
						\item  \textbf{Cilj:} Pregled podataka nestalih osoba
						\item  \textbf{Sudionici:} Baza podataka
						\item  \textbf{Preduvjet:} Korisnik je prijavljen u sustav
						\item  \textbf{Opis osnovnog tijeka:}
						
						\item[] \begin{packed_enum}
	
							\item Korisnik odabire ekran "Nestali"
							\item Aplikacija daje prikaz podataka nestalih osoba
						\end{packed_enum}
						
					
					\end{packed_item}
					
				\noindent \underbar{\textbf{UC4 - Prijava nestale osobe}}
					\begin{packed_item}
	
						\item \textbf{Glavni sudionik: }Korisnik
						\item  \textbf{Cilj:} Prijaviti nestanak osobe
						\item  \textbf{Sudionici:} Baza podataka
						\item  \textbf{Preduvjet:} -
						\item  \textbf{Opis osnovnog tijeka:}
						
						\item[] \begin{packed_enum}
	
							\item Korisnik unosi podatke o nestaloj osobi u sustav
							
						\end{packed_enum}
						
						\item  \textbf{Opis mogućih odstupanja:}
						
						\item[] \begin{packed_item}
	
							\item[1.a] Prijava već prijavljene nestale osobe
							\item[] \begin{packed_enum}
								
								\item Sustav obavještava korisnika da je osoba već prijavljena kao nestala i vraća ga na stranicu "Nestali"
								
								
							\end{packed_enum}
							
							
						\end{packed_item}
					\end{packed_item}
					\noindent \underbar{\textbf{UC5 - Komentiranje prijave nestale osobe}}
					\begin{packed_item}
	
						\item \textbf{Glavni sudionik: }Korisnik
						\item  \textbf{Cilj:} Komentiranje prijave nestale osobe
						\item  \textbf{Sudionici:} Baza podataka
						\item  \textbf{Preduvjet:} -
						\item  \textbf{Opis osnovnog tijeka:}
						
						\item[] \begin{packed_enum}
	
							\item Korisnik piše komentar na prijavu o nestaloj osobi
							
						\end{packed_enum}
						
					
					\end{packed_item}
					\noindent \underbar{\textbf{UC6 - Zaključavanje prijave }}
					\begin{packed_item}
	
						\item \textbf{Glavni sudionik: }Spasioc ili voditelj operacije
						\item  \textbf{Cilj:} Zaključavanje prijave o nestaloj osobi
						\item  \textbf{Sudionici:} Baza podataka
						\item  \textbf{Preduvjet:} Osoba je pronađena
						\item  \textbf{Opis osnovnog tijeka:}
						
						\item[] \begin{packed_enum}
	
							\item Spasioc ili voditelj operacije odabire ekran "Nestali"
							\item Spasioc ili voditelj operacije zaključava prijavu o nestaloj osobi
							\item Status prijave mijenja se u "zaključana".
							\item Prijava se više ne prikazuje u sustavu
							
						\end{packed_enum}
						
					
					\end{packed_item}
					\noindent \underbar{\textbf{UC7 - Brisanje prijave}}
					\begin{packed_item}
	
						\item \textbf{Glavni sudionik: }Voditelj operacije
						\item  \textbf{Cilj:} Uklanjanje prijave o nestaloj osobi
						\item  \textbf{Sudionici:} Baza podataka
						\item  \textbf{Preduvjet:} Voditelj operacije je prijavljen u sustav
						\item  \textbf{Opis osnovnog tijeka:}
						
						\item[] \begin{packed_enum}
	
							\item Voditelj operacije odabire ekran "Nestali"
							\item Voditelj operacije odabire gumb "Obriši prijavu"
							\item Prijava o nestaloj osobi briše se iz baze podataka
							
						\end{packed_enum}
						
					
					\end{packed_item}
					\noindent \underbar{\textbf{UC8 - Brisanje komentara}}
					\begin{packed_item}
	
						\item \textbf{Glavni sudionik: }Voditelj operacije
						\item  \textbf{Cilj:} Brisanje komentara sa prijave o nestaloj osobi
						\item  \textbf{Sudionici:} Baza podataka
						\item  \textbf{Preduvjet:} Postoji barem 1 komentar na prijavi o nestaloj osobi
						\item  \textbf{Opis osnovnog tijeka:}
						
						\item[] \begin{packed_enum}
	
							\item Voditelj operacije odabire željenu prijavu o nestaloj osobi
							\item Voditelj operacije odabire gumb "Obriši komentar" za željeni komentar
							
						\end{packed_enum}
						
					
					\end{packed_item}
					\noindent \underbar{\textbf{UC9 - Prikaz karte}}
					\begin{packed_item}
	
						\item \textbf{Glavni sudionik: }Korisnik
						\item  \textbf{Cilj:} Prikaz područja na karti
						\item  \textbf{Sudionici:} Baza podataka
						\item  \textbf{Preduvjet:} Korisnik je prijavljen u sustav
						\item  \textbf{Opis osnovnog tijeka:}
						
						\item[] \begin{packed_enum}
	
							\item Nakon prijave korisniku je vidljiva karta sa označenim regijama i blokovima svih operacija
							\item 
							\item $<$opis korak tri$>$
							\item $<$opis korak četiri$>$
							\item $<$opis korak pet$>$
						\end{packed_enum}
						
						\item  \textbf{Opis mogućih odstupanja:}
						
						\item[] \begin{packed_item}
	
							\item[2.a] $<$opis mogućeg scenarija odstupanja u koraku 2$>$
							\item[] \begin{packed_enum}
								
								\item $<$opis rješenja mogućeg scenarija korak 1$>$
								\item $<$opis rješenja mogućeg scenarija korak 2$>$
								
							\end{packed_enum}
							\item[2.b] $<$opis mogućeg scenarija odstupanja u koraku 2$>$
							\item[3.a] $<$opis mogućeg scenarija odstupanja  u koraku 3$>$
							
						\end{packed_item}
					\end{packed_item}
					\noindent \underbar{\textbf{UC10 - Stvaranje operacije}}
					\begin{packed_item}
	
						\item \textbf{Glavni sudionik: }Voditelj operacije
						\item  \textbf{Cilj:} Definiranje područja koje je potrebno pretražiti
						\item  \textbf{Sudionici:} Baza podataka
						\item  \textbf{Preduvjet:} Voditelj operacije je prijavljen u sustav
						\item  \textbf{Opis osnovnog tijeka:}
						
						\item[] \begin{packed_enum}
	
							\item Voditelj operacije odabire ekran "Karta"  
							\item Voditelj operacije odabire gumb "Stvori novu operaciju"
							\item Voditelj operacije unosi potrebne podatke o operaciji: ime operacije i područje (regije i blokove) koje je potrebno pretražiti
							\item Podaci o stvorenoj operaciji spremaju se u bazu podataka
						\end{packed_enum}
						
						\item  \textbf{Opis mogućih odstupanja:}
						
						\item[] \begin{packed_item}
	
							\item[2.a] Stvorena je operacija s već postojećim imenom
							\item[] \begin{packed_enum}
								
								\item Sustav onemogućuje stvaranje nove opracije, te javlja grešku.

							\end{packed_enum}
							
						\end{packed_item}
					\end{packed_item}
				\subsubsection{Dijagrami obrazaca uporabe}
					
					\textit{Prikazati odnos aktora i obrazaca uporabe odgovarajućim UML dijagramom. Nije nužno nacrtati sve na jednom dijagramu. Modelirati po razinama apstrakcije i skupovima srodnih funkcionalnosti.}
				\eject		
				
			\subsection{Sekvencijski dijagrami}
				
				\textbf{\textit{dio 1. revizije}}\\
				
				\textit{Nacrtati sekvencijske dijagrame koji modeliraju najvažnije dijelove sustava (max. 4 dijagrama). Ukoliko postoji nedoumica oko odabira, razjasniti s asistentom. Uz svaki dijagram napisati detaljni opis dijagrama.}
				\eject
	
		\section{Ostali zahtjevi}
		
			\textbf{\textit{dio 1. revizije}}\\
		 
			 \textit{Nefunkcionalni zahtjevi i zahtjevi domene primjene dopunjuju funkcionalne zahtjeve. Oni opisuju \textbf{kako se sustav treba ponašati} i koja \textbf{ograničenja} treba poštivati (performanse, korisničko iskustvo, pouzdanost, standardi kvalitete, sigurnost...). Primjeri takvih zahtjeva u Vašem projektu mogu biti: podržani jezici korisničkog sučelja, vrijeme odziva, najveći mogući podržani broj korisnika, podržane web/mobilne platforme, razina zaštite (protokoli komunikacije, kriptiranje...)... Svaki takav zahtjev potrebno je navesti u jednoj ili dvije rečenice.}
			 
			 
			 
	