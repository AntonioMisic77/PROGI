

\chapter{Zaključak i budući rad}
		
		Naš zadatak bio je razvoj web aplikacije koja služi za pronalaženje nestalih osoba te implementacija karte kako bi se olakšao posao spasiocima. Za samu izradnju karte bili su odgovorni kartografi dok su voditelji bili zaduženi za održavanje opearcija te određivanja regija koje je potrebno kartografirati. Zadani cilj smo postigli nakon 13 tjedana timskog rada.

        Prvi dio projekta se sastojao od okupljanja i upoznavanja članova tima. Na tom sastanku članovi su se podijelili i međusobno dogovorili tko je zadužen za koji dio projekta. Tijekom ove faze većina se članova koncentrirala na pisanje i obradu dokumentacije, što je bilo učinjeno bez većih poteškoća. Na kasnijim sastancima raspravljalo se o tome kako bi \textit{baza podataka}, \textit{frontend} i \textit{backend} trebali izgledati. Najveći dio tijekom ove faze projekta su bili deployment te implementacija slike tijekom registracije korisnika.

        Drugi dio projekta se manje oslanjao na dokumentaciju te više na samu implementaciju \textit{use-caseova}. Pošto je već nekoliko članova imalo iskustva s \textit{backendom}, lako su svoje znanje prenijeli i na ostale članove kojima je zadaća bila programiranje i implementiranje funkcija web aplikacije. Takve upute su nam pomogle da se svi uskladimo kako ne bi kasnije došlo do komplikacija tijekom pisanja koda. Kod dokumentacije je još bilo potrebno dovršiti ostale UML dijagrame, ispitivanje programskog rješenja te upute za puštanje u pogon. Sa \textit{frontendom} je bilo manje problema nego s \textit{backendom}, s time da je najteži dio implementacije ove faze projekta bila sama karta kojom se služe kartografi, voditelji i spasioci.

        Za komunikaciju među članovima služili smo se aplikacijom \textit{WhatsApp} gdje smo često postavljali pitanja i pomagali drugima oko dokumentacije ili manjih problema u vezi projekta. Za komunikaciju tijekom rješavanja većih problema projekta kao što su bili karta ili implementacija kompleksnijih funkcionalnosti koristili smo aplikaciju \textit{Discord}. Budući rad bi mogao uključivati mobilno proširenje kako bi se povećala pristupačnost i pomoć kod pronalaženja nestalih osoba.

        Izrada ovog projektnog zadatka dobro je doprinijela razvoju vještina članovima tima koji se prvi put upoznavaju s ovakvim zadatokm dok su već vješti članovi iskoristili ovu priliku kako bi usavršili svoje sposobnosti te pomogli drugima tako što su im prenijeli svoje znanje o razvoju web aplikacija.
\begin{comment}		
		 \textit{U ovom poglavlju potrebno je napisati osvrt na vrijeme izrade projektnog zadatka, koji su tehnički izazovi prepoznati, jesu li riješeni ili kako bi mogli biti riješeni, koja su znanja stečena pri izradi projekta, koja bi znanja bila posebno potrebna za brže i kvalitetnije ostvarenje projekta i koje bi bile perspektive za nastavak rada u projektnoj grupi.}

		 \textit{Potrebno je točno popisati funkcionalnosti koje nisu implementirane u ostvarenoj aplikaciji.}
		
\end{comment}


